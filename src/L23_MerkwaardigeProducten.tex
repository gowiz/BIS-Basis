% Basiscursus Wiskunde - Lespakket 8
% Les 23. Merkwaardige producten
\documentclass[a4paper,12pt]{article}

\usepackage[margin=2.5cm]{geometry}
\usepackage[dutch]{babel}
\usepackage[T1]{fontenc}
\usepackage[utf8]{inputenc}
\usepackage{amsmath,amssymb}
\usepackage{enumitem}
\usepackage{multicol}
\usepackage{siunitx}
\usepackage{hyperref}

\sisetup{locale=DE,output-decimal-marker={,}}
\setlist[enumerate]{label=\alph*), leftmargin=2em}

\title{Basiscursus Wiskunde\\\large Les 23. Merkwaardige producten}
\date{}
\author{}

\begin{document}
\maketitle

\section*{Inleiding}
In de algebra komen zeer veel producten met veeltermen voor. Sommige van deze producten kunnen op een andere wijze berekend worden dan de klassieke vermenigvuldigingen zoals aangeleerd in les~22.

Omdat ze telkens iets merkwaardigs bevatten, noemen we ze \emph{merkwaardige producten}. Het zijn producten die een belangrijke rol spelen in de wiskunde. Daarom leren we hun resultaat correct uit het hoofd opschrijven. Om ze beter te onderscheiden van gewone oefeningen, gebruiken we blokletters voor de modelformules.

\section{Product van twee tweetermen met een paar gelijke en een paar tegengestelde termen}
In het product \((A+B)(A-B)\) stelt \(A\) het paar gelijke en \(B\) het paar tegengestelde termen voor. We werken het product uit:
\[
(A+B)(A-B)=A^2-AB+AB-B^2=A^2-B^2.
\]
Doordat de tegengestelde termen wegvallen, blijven alleen de kwadraten over.

\paragraph*{Besluit (uit het hoofd te leren):}
\[
\boxed{(A+B)(A-B)=A^2-B^2}
\]

\paragraph*{Voorbeeld 1}
\((3x+5)(3x-5)=9x^2-25\)\\
\(A=3x\Rightarrow A^2=(3x)^2=9x^2\), \quad \(B=5\Rightarrow B^2=5^2=25\).

\paragraph*{Voorbeeld 2}
\(({-}2a+7b)(2a+7b)=49b^2-4a^2\)\\
\(A=7b\Rightarrow A^2=49b^2\), \quad \(B=2a\Rightarrow B^2=4a^2\).\\
Merk op dat de gelijke term \(A\) niet altijd als eerste in de tweetermen voorkomt.

\paragraph*{Voorbeeld 3}
\(({-}x^3-y^3)(x^3-y^3)=y^6-x^6\)\\
\(A=y^3\Rightarrow A^2=y^6\), \quad \(B=x^3\Rightarrow B^2=x^6\).

\paragraph*{Nog enkele voorbeelden}
\[
\begin{aligned}
({-}x-y)({-}x+y) &= x^2-y^2,\\
(5x^3+7y)(5x^3-7y) &= 25x^6-49y^2,\\
\Bigl(\tfrac{a}{3}+a^2\Bigr)\Bigl(\tfrac{a}{3}-a^2\Bigr) &= \tfrac{1}{9}a^2-a^4,\\
(ab+c)(c-ab) &= c^2-a^2b^2,\\
(2a^2-3b^3c^4)(3b^3c^4+2a^2) &= 4a^4-9b^6c^8,\\
({-}x^3+6)(6+x^3) &= 36-x^6.
\end{aligned}
\]

Bij sommige vermenigvuldigingen met getallen kunnen we dit merkwaardig product ook toepassen:
\[
\begin{aligned}
51\cdot 49 &= (50+1)(50-1)=2500-1=2499,\\
74\cdot 86 &= (80-6)(80+6)=6400-36=6364.
\end{aligned}
\]

\subsection*{A.C.O.\ 1}
\emph{Schrijf onmiddellijk het resultaat (onderstreep de gelijke termen).}
\begin{multicols}{2}
  \begin{enumerate}
  \item \((x+3)(x-3)=\;\ldots\)
  \item \((x^3+1)(x^3-1)=\;\ldots\)
  \item \(({-}a+3)({-}3-a)=\;\ldots\)
  \item \(({-}2x-y)(y-2x)=\;\ldots\)
  \item \((x^6-7y^5)(7y^5+x^6)=\;\ldots\)
  \item \((x^2-\tfrac{1}{2})(x^2+\tfrac{1}{2})=\;\ldots\)
  \item \((2a^3b-5c)(5c+2a^3b)=\;\ldots\)
  \item \((p-0{,}4)(0{,}4+p)=\;\ldots\)
  \item \((2x^2-3xy)({-}3xy-2x^2)=\;\ldots\)
  \item \(21\cdot 19=\;\ldots=\;\ldots=\;\ldots\)
  \item \(102\cdot 98=\;\ldots=\;\ldots=\;\ldots\)
  \item \(45\cdot 55=\;\ldots=\;\ldots=\;\ldots\)
  \end{enumerate}
\end{multicols}

\paragraph*{Opgelet!} De formule wordt soms meermaals toegepast.

\paragraph*{Voorbeeld 1}
\[
\begin{aligned}
(2x-3y)(2x+3y)(4x^2+9y^2)
  &=(4x^2-9y^2)(4x^2+9y^2)\\
  &=16x^4-81y^4.
\end{aligned}
\]

\paragraph*{Voorbeeld 2}
\[
\begin{aligned}
(x^4+81)(x+3)(x^2+9)(x-3)
  &=(x^4+81)(x^2+9)(x^2-9)\\
  &=(x^4+81)(x^4-81)\\
  &=x^8-6561.
\end{aligned}
\]

\subsection*{A.C.O.\ 2}
\emph{Gebruik het merkwaardig product om te berekenen:}
\begin{enumerate}
  \item \((2a+3)(2a-3)(4a^2+9)\)
  \item \((a^4+b^4)(a^2+b^2)(a+b)(a-b)\)
  \item \((a+2)(a^4+16)(a^2+4)(a-2)\)
\end{enumerate}

\section{Het kwadraat van een tweeterm}
We willen uitwerken: \((A+B)^2\). Dit is de kortere schrijfwijze voor \((A+B)\cdot(A+B)\). Volgens de distributieve eigenschap:
\begin{align*}
  (A+B)(A+B) & =A^2+AB+AB+B^2 \\
             & =A^2+2AB+B^2.
\end{align*}

\paragraph*{Formule (uit het hoofd te leren):}
\[
\boxed{(A+B)^2=A^2+2AB+B^2}
\]
\paragraph*{Aandacht!} \((A+B)^2\neq A^2+B^2\).

\paragraph*{Voorbeeld 1}
\((3a+5b)^2=9a^2+30ab+25b^2\).

\paragraph*{Voorbeeld 2}
\((3x-4)^2=9x^2-24x+16\).

\paragraph*{Voorbeeld 3}
\(({-}5a-4b)^2=25a^2+40ab+16b^2\).

\paragraph*{Voorbeeld 4}
\(({-}2k+3)^2=4k^2-12k+9\).

\paragraph*{Tekenregel}
In \(A^2+2AB+B^2\) hebben \(A^2\) en \(B^2\) altijd een plusteken. Als \(A\) en \(B\) hetzelfde teken hebben, dan is \(2AB>0\); anders \(2AB<0\).

\paragraph*{Nog enkele voorbeelden}
\[
\begin{aligned}
(4x^2-9x)^2 &= 16x^4-72x^3+81x^2,\\
(3x+7)^2 &= 9x^2+42x+49,\\
\Bigl(\tfrac{1}{3}a+\tfrac{3}{4}\Bigr)^2 &= \tfrac{1}{9}a^2+\tfrac{1}{2}a+\tfrac{9}{16},\\
(5p^2-6p)^2 &= 25p^4-60p^3+36p^2,\\
(0{,}6+5t^4)^2 &= 0{,}36+6t^4+25t^8,\\
({-}4a^6+7a^5)^2 &= 16a^{12}-56a^{11}+49a^{10},\\
({-}2a-5b)^2 &= 4a^2+20ab+25b^2.
\end{aligned}
\]

Ook kwadraten van getallen kunnen zo berekend worden:
\[
\begin{aligned}
98^2&=(100-2)^2=10000-400+4=9604,\\
53^2&=(50+3)^2=2500+300+9=2809,\\
65^2&=(60+5)^2=3600+600+25=4225\\
     &=(70-5)^2=4900-700+25=4225.
\end{aligned}
\]

\subsection*{A.C.O.\ 3}
\emph{Schrijf onmiddellijk het resultaat.}
\begin{multicols}{2}
  \begin{enumerate}
  \item \((7x+6)^2=\;\ldots\)
  \item \(({-}5x-3y)^2=\;\ldots\)
  \item \((a^5-1)^2=\;\ldots\)
  \item \(({-}3x^2+7x)^2=\;\ldots\)
  \item \((x^5-\tfrac{1}{3})^2=\;\ldots\)
  \item \((2a+3b^2)^2=\;\ldots\)
  \item \(({-}x-2y)^2=\;\ldots\)
  \item \(({-}2a^2b-4c)^2=\;\ldots\)
  \item \(61^2=\;\ldots=\;\ldots=\;\ldots\)
  \item \(27^2=\;\ldots=\;\ldots=\;\ldots\)
  \end{enumerate}
\end{multicols}
\subsection*{Toepassingen op beide merkwaardige producten}
\paragraph*{Toepassing 1}
\((a-2)(a+2)(a^2-4)=(a^2-4)^2=a^4-8a^2+16.\)

\paragraph*{Toepassing 2}
\((4x^2-9y^2)(2x-3y)(2x+3y)=(4x^2-9y^2)^2=16x^4-72x^2y^2+81y^4.\)

\paragraph*{Toepassing 3}
\((5x+4y+3z)(5x+4y-3z)=[(5x+4y)+3z][(5x+4y)-3z]\)
\[
=(5x+4y)^2-(3z)^2=25x^2+40xy+16y^2-9z^2.
\]

\paragraph*{Toepassing 4}
\((2x+5-3x^2)(2x-5+3x^2)=[2x+(5-3x^2)][2x-(5-3x^2)]\)
\[
=(2x)^2-(5-3x^2)^2=4x^2-(25-30x^2+9x^4)=-9x^4+34x^2-25.
\]

\subsection*{A.C.O.\ 4}
\emph{Uitwerken zonder bijkomende uitleg:}
\begin{enumerate}
  \item \((x+1)(x-1)(x^2+1)(x^4+1)(x^8-1)\)
  \item \((a^2+a+1)(a^2-a+1)\)
  \item \((5x+4y+3z)(5x-4y-3z)\)
\end{enumerate}

\section{Het kwadraat van een veelterm}
Uitwerking:
\[
\begin{aligned}
(a+b+c)^2&=(a+b+c)(a+b+c)\\
&=a^2+b^2+c^2+2ab+2ac+2bc.
\end{aligned}
\]

\paragraph*{Formule}
\[
\boxed{(A+B+C)^2=A^2+B^2+C^2+2AB+2AC+2BC}
\]
Het kwadraat van een veelterm bestaat uit de som van alle mogelijke kwadraten en alle mogelijke dubbele producten.

\paragraph*{Voorbeeld 1}
\((2a-3b+4c)^2=4a^2+9b^2+16c^2-12ab+16ac-24bc\).

\paragraph*{Voorbeeld 2}
\((2x^3-3x^2+x-5)^2\)
\[
=4x^6-12x^5+13x^4-26x^3+31x^2-10x+25.
\]

\subsection*{A.C.O.\ 5}
\emph{Werk uit:}
\begin{enumerate}
  \item \((2x^2-x+3)^2\)
  \item \((a^3-2a^2+4a-3)^2\)
\end{enumerate}

\section{Derdemacht van een tweeterm}
Omdat \( (a+b)^3=(a+b)(a+b)^2=(a+b)(a^2+2ab+b^2)\), volgt na uitwerken:
\[
\boxed{(A+B)^3=A^3+3A^2B+3AB^2+B^3}.
\]

\paragraph*{Uitgewerkte voorbeelden}
\[
\begin{aligned}
(5x+2y)^3&=125x^3+150x^2y+60xy^2+8y^3,\\
(2a-b)^3&=8a^3-12a^2b+6ab^2-b^3,\\
({-}4a+3b^2)^3&={-}64a^3+144a^2b^2-108ab^4+27b^6,\\
({-}a-2)^3&={-}a^3-6a^2-12a-8.
\end{aligned}
\]

\paragraph*{Belangrijke opmerking (tekenregel)}
Alleen de tekens:
\[
(+\ +)^3=+\ +\ +\ +,\quad (+\ -)^3=+\ -\ +\ -,\\
(-\ +)^3=-\ +\ -\ +,\quad (-\ -)^3=-\ -\ -\ -.
\]

\paragraph*{Praktische werkwijze}
\((3a-4b)^3=\;+\,-\,+\,-\;\Rightarrow\; +27a^3-108a^2b+144ab^2-64b^3.\)

Nog:
\(({-}2p-t)^3=-8p^3-12p^2t-6pt^2-t^3\), \quad
\(({-}5x+2y)^3={-}125x^3+150x^2y-60xy^2+8y^3\),\\
\((k+6)^3=k^3+18k^2+108k+216\).

\subsection*{A.C.O.\ 6}
\emph{Werk uit:}
\begin{enumerate}
  \item \((5x-6)^3\)
  \item \(({-}x^3+x^2)^3\)
  \item \(({-}4x^2-3xy)^3\)
  \item \((a^2+b^2)^3\)
  \item \((2a-1)^3\)
\end{enumerate}

\section{Product van een tweeterm met aangepaste drieterm}
We berekenen:
\[
(a+b)(a^2-ab+b^2)=a^3+b^3.
\]

\paragraph*{Formule}
\[
\boxed{(A+B)(A^2-AB+B^2)=A^3+B^3}
\]
De aangepaste drieterm bevat de kwadraten van beide termen van de tweeterm \emph{min} het product van die beide termen.

\paragraph*{Regel (tekeninzicht)}
De uitkomst heeft dezelfde tekens als de tweeterm; voor \(AB\) neem je \(-\) als \(A,B\) hetzelfde teken hebben en \(+\) als ze verschillend zijn.

\paragraph*{Uitgewerkte voorbeelden}
\[
\begin{aligned}
(2x+3y)(4x^2-6xy+9y^2)&=8x^3+27y^3,\\
(x-2)(x^2+2x+4)&=x^3-8,\\
({-}3a-4b)(9a^2-12ab+16b^2)&={-}27a^3-64b^3,\\
({-}5p+1)(25p^2+5p+1)&={-}125p^3+1.
\end{aligned}
\]

\subsection*{A.C.O.\ 7}
\emph{Schrijf onmiddellijk de uitkomst (vul waar nodig de aangepaste drieterm in).}
\begin{enumerate}
  \item \((x-3)(x^2+3x+9)=\;\ldots\)
  \item \((3x+5y)(\underline{\hspace{3cm}})=\;\ldots\) \quad \emph{(Vul ook de aangepaste drieterm in.)}
  \item \((a^2-6b)(\underline{\hspace{3cm}})=\;\ldots\)
  \item \((a^4+2a^3)(\underline{\hspace{3cm}})=\;\ldots\)
  \item \(({-}4x-7y)(\underline{\hspace{3cm}})=\;\ldots\)
\end{enumerate}

\section*{OPLOSSINGEN A.C.O.}
\paragraph*{1.}
\begin{enumerate}
  \item \(x^2-9\)
  \item \(x^6-1\)
  \item \(a^2-9\)
  \item \(4x^2-y^2\)
  \item \(x^{12}-49y^{10}\)
  \item \(x^4-\tfrac{1}{4}\)
  \item \(4a^6b^2-25c^2\)
  \item \(p^2-0{,}16\)
  \item \(9x^2y^2-4x^4\)
  \item \((20+1)(20-1)=400-1=399\)
  \item \((100+2)(100-2)=10000-4=9996\)
  \item \((50-5)(50+5)=2500-25=2475\)
\end{enumerate}

\paragraph*{2.}
\begin{enumerate}
  \item \((4a^2-9)(4a^2+9)=16a^4-81\)
  \item \((a^4+b^4)(a^2+b^2)(a^2-b^2)=(a^4+b^4)(a^4-b^4)=a^8-b^8\)
  \item \((a^2-4)(a^4+16)(a^2+4)=(a^4-16)(a^4+16)=a^8-256\)
\end{enumerate}

\paragraph*{3.}
\begin{enumerate}
  \item \(49x^2+84x+36\)
  \item \(25x^2+30xy+9y^2\)
  \item \(a^{10}-2a^5+1\)
  \item \(9x^4-42x^3+49x^2\)
  \item \(x^{10}-\tfrac{2}{3}x^5+\tfrac{1}{9}\)
  \item \(4a^2+12ab^2+9b^4\)
  \item \(x^2+4xy+4y^2\)
  \item \(4a^4b^2+16a^2bc+16c^2\)
  \item \((60+1)^2=3600+120+1=3721\)
  \item \((30-3)^2=900-180+9=729\)
\end{enumerate}

\paragraph*{4.}
\begin{enumerate}
  \item \((x^8-1)^2=x^{16}-2x^8+1\)
  \item \([(a^2+1)+a][(a^2+1)-a]=(a^2+1)^2-a^2=a^4+a^2+1\)
  \item \([5x+(4y+3z)][5x-(4y+3z)]=25x^2-(16y^2+24yz+9z^2)\\=25x^2-16y^2-24yz-9z^2\)
\end{enumerate}

\paragraph*{5.}
\begin{enumerate}
  \item \(4x^4-4x^3+13x^2-6x+9\)
  \item \(a^6-4a^5+12a^4-22a^3+28a^2-24a+9\)
\end{enumerate}

\paragraph*{6.}
\begin{enumerate}
  \item \(125x^3-450x^2+540x-216\)
  \item \({-}x^9+3x^8-3x^7+x^6\)
  \item \({-}64x^6-144x^5y-108x^4y^2-27x^3y^3\)
  \item \(a^6+3a^4b^2+3a^2b^4+b^6\)
  \item \(8a^3-12a^2+6a-1\)
\end{enumerate}

\paragraph*{7.}
\begin{enumerate}
  \item \(x^3-27\)
  \item \((3x+5y)(9x^2-15xy+25y^2)=27x^3+125y^3\)
  \item \((a^2-6b)(a^4+6a^2b+36b^2)=a^6-216b^3\)
  \item \((a^4+2a^3)(a^8-2a^7+4a^6)=a^{12}+8a^9\)
  \item \(({-}4x-7y)(16x^2-28xy+49y^2)={-}64x^3-343y^3\)
\end{enumerate}

\end{document}

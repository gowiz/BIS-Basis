% Basiscursus Wiskunde - Lespakket 1
% Les 1. Metriek stelsel - Vlakken en figuren
\documentclass[a4paper,12pt]{article}

\usepackage[margin=2.5cm]{geometry}
\usepackage[dutch]{babel}
\usepackage[T1]{fontenc}
\usepackage[utf8]{inputenc}
\usepackage{amsmath,amssymb}
\usepackage{enumitem}
\usepackage{booktabs}
\usepackage{array}
\usepackage{multicol}
\usepackage{siunitx}
\usepackage{hyperref}
\usepackage{graphicx}
\usepackage{etoolbox}
\usepackage{titlesec}

% Lijsten
\setlist{itemsep=0.5em, topsep=0.5em, parsep=0pt, partopsep=0pt}
\AtBeginEnvironment{multicols}{%
  \setlist[enumerate]{%
    before=\vspace*{-3.6em},
    after=\vspace*{-0.5em}
  }%
}
\setlist[enumerate,1]{label=(\alph*)}

% siunitx: decimale komma en nette groepering
\sisetup{
  locale = DE,
  detect-weight = true,
  detect-family = true,
  group-separator = {\,},
  group-minimum-digits = 4
}

\DeclareSIUnit{\carat}{karaat}
\DeclareSIUnit{\are}{are}
\DeclareSIUnit{\litre}{l}

\title{Basiscursus Wiskunde\\\large Les 1. Metriek stelsel -- Vlakken en figuren}
\date{}
\author{}

\begin{document}
\maketitle

\section{De lengtematen}

\subsection{Even terug in de tijd}
Oorspronkelijk werden de meeste lengtematen ontleend aan de afmetingen van het menselijk lichaam. Je hoorde wellicht reeds van de oude lengtematen \emph{voet}, \emph{el} (van elleboog) en \emph{duim}. De Engelse maat \emph{yard} was bepaald als de afstand van de neuspunt van koning Henry~I tot de top van de duim van zijn uitgestrekte arm. Je kan je wel voorstellen dat deze maten niet nauwkeurig waren en allerminst bruikbaar in de handel.

Om dit probleem op te lossen werd, rond 1790, het metriek stelsel ingevoerd. Als lengte-eenheid werd het veertig miljoenste deel van de omtrek van de aarde genomen en dit werd de \emph{meter} genoemd. In het \emph{Bureau International des Poids et Mesures} te Sèvres wordt een staaf van platina-iridium als standaardmaat bewaard.

\subsection{Overzicht van de lengtematen}
Alle onderdelen en veelvouden van de meter worden opgebouwd volgens het tientallig stelsel, d.w.z.\ dat elke lengtemaat steeds het tienvoud is van de volgende.

\paragraph{Omzettingstabel LENGTEMATEN}
\renewcommand{\arraystretch}{1.2}
\begin{center}
\setlength{\fboxsep}{4pt}\setlength{\fboxrule}{0.5pt}%
\begin{tabular}{*{7}{c}}
  \multicolumn{3}{c}{\textbf{Veelvouden}} & \textbf{Eenheid} & \multicolumn{3}{c}{\textbf{Onderdelen}}\\
  \midrule
\SI{1}{\kilo\metre} & \SI{1}{\hecto\metre} & \SI{1}{\deca\metre} &  & \SI{1}{\deci\metre} & \SI{1}{\centi\metre} & \SI{1}{\milli\metre}\\
= & = & = & \textbf{\SI{1}{\metre}} & = & = & =\\
\SI{1000}{\metre} & \SI{100}{\metre} & \SI{10}{\metre} & & \SI{0.1}{\metre} & \SI{0.01}{\metre} & \SI{0.001}{\metre}
\end{tabular}%
\end{center}

Verder onderscheiden we nog:
\begin{itemize}
\item als veelvoud:
  \begin{itemize}
  \item \SI{1}{\mega\metre} $= \SI{1e6}{\metre}$.
  \item \emph{lichtjaar} $\approx \SI{9.5e15}{\metre}$; dit wordt gebruikt om afstanden tussen sterren te berekenen.
  \end{itemize}
\item als onderdeel:
  \begin{itemize}
  \item \SI{1}{\micro\metre} $= \SI{1e-6}{\metre} = \SI{0.001}{\milli\metre}$.
  \end{itemize}
\end{itemize}

\subsection{Herleiden}
Lengtematen herleiden betekent dat we moeten nagaan hoeveel eenheden van de ene eenheid in de andere vervat zijn.

\subsection*{Bij omzetting naar een kleinere lengte-eenheid:}

\subsubsection*{Voorbeeld 1}
van \si{\deci\metre} $\to$ \si{\centi\metre}: \quad \SI{1}{\deci\metre} = \SI{10}{\centi\metre}\\[0.4em]
van \si{\hecto\metre} $\to$ \si{\deci\metre}: \quad \SI{1}{\hecto\metre} = \SI{10}{\deca\metre} = \SI{100}{\metre} = \SI{1000}{\deci\metre}

Praktisch: van \si{\hecto\metre} naar \si{\deci\metre} is 3 rangen. Je verschuift de komma 3 rangen naar rechts of plaatst 3 nullen bij. Dus \SI{1}{\hecto\metre} = \SI{1000}{\deci\metre}.

\subsubsection*{Voorbeeld 2}
\(\SI{18}{\hecto\metre}=\ldots\,\si{\centi\metre}\)

\[
  \si{\hecto\metre}\xrightarrow[1]{}\si{\deca\metre}\xrightarrow[2]{}\si{\metre}\xrightarrow[3]{}\si{\deci\metre}\xrightarrow[4]{}\si{\centi\metre}
\]
4 rangen naar rechts: 4 nullen bijplaatsen. Dus \(\SI{18}{\hecto\metre}=\SI{180000}{\centi\metre}\).

\subsection*{Bij omzetting naar een grotere eenheid verplaatsen we de komma naar links.}

\subsubsection*{Voorbeeld 1}
\(\SI{4500}{\metre}=\ldots\,\si{\kilo\metre}\)
\[
\si{\kilo\metre}\xleftarrow[3]{}\si{\hecto\metre}\xleftarrow[2]{}\si{\deca\metre}\xleftarrow[1]{}\si{\metre}
\]
3 rangen naar links: komma 3 plaatsen naar links. Dus \(\SI{4500}{\metre}=\SI{4.5}{\kilo\metre}\).

\subsubsection*{Voorbeeld 2}
\(\SI{0.2}{\deci\metre}=\ldots\,\si{\deca\metre}\)
\[
\si{\deca\metre}\xleftarrow[2]{}\si{\metre}\xleftarrow[1]{}\si{\deci\metre}
\]
2 rangen naar links: komma 2 plaatsen naar links. Dus \(\SI{0.2}{\deci\metre}=\SI{0.002}{\deca\metre}\).

\subsection*{A.C.O.~1}
\begin{multicols}{2}
\begin{enumerate}
  \item \(\SI{27.5}{\deci\metre}=\ldots\,\si{\milli\metre}\)
  \item \(\SI{5}{\hecto\metre}=\ldots\,\si{\deci\metre}\)
  \item \(\SI{5.66}{\metre}=\ldots\,\si{\milli\metre}\)
  \item \(\SI{0.02}{\metre}=\ldots\,\si{\centi\metre}\)
  \item \(\SI{126}{\deci\metre}=\ldots\,\si{\kilo\metre}\)
  \item \(\SI{5.3}{\centi\metre}=\ldots\,\si{\metre}\)
  \item \(\SI{3}{\centi\metre}=\ldots\,\si{\deca\metre}\)
  \item \(\SI{0.04}{\deca\metre}=\ldots\,\si{\hecto\metre}\)
\end{enumerate}
\end{multicols}


\subsection{Als er verschillende eenheden in de opgave staan}
Herleid alle getallen naar dezelfde eenheid.

\paragraph{Voorbeeld} \(\SI{1}{\metre}=\SI{7}{\deci\metre}+\ldots\,\si{\centi\metre}\). We herleiden naar \si{\centi\metre}: \(\SI{100}{\centi\metre}=\SI{70}{\centi\metre}+\SI{30}{\centi\metre}\).

\subsection*{A.C.O.~2}
\begin{enumerate}
  \item \(\SI{1}{\kilo\metre}-\SI{1}{\metre}=\ldots\,\si{\metre}\)
  \item \(\SI{4}{\centi\metre}+\ldots=\SI{1}{\deci\metre}\)
  \item \(\SI{35}{\centi\metre}+\ldots=\SI{4}{\deci\metre}\)
  \item \(\SI{4}{\metre}-\SI{7}{\deci\metre}=\ldots\,\si{\centi\metre}\)
\end{enumerate}

\section{De massa}
We gebruiken het begrip ``massa'' in plaats van ``gewicht''. Volgens de principes van de fysica is massa een juister begrip.

\subsection{Overzicht van de massa's}
De hoofdeenheid van massa is de gram. Ook de massa's worden uitgedrukt in een tientallig stelsel.

\paragraph{Omzettingstabel MASSA}
\renewcommand{\arraystretch}{1.2}
\begin{center}
\setlength{\fboxsep}{4pt}\setlength{\fboxrule}{0.5pt}%
\begin{tabular}{*{7}{c}}
  \multicolumn{3}{c}{\textbf{Veelvouden}} & \textbf{Eenheid} & \multicolumn{3}{c}{\textbf{Onderdelen}}\\
  \midrule
\SI{1}{\kilo\gram} & \SI{1}{\hecto\gram} & \SI{1}{\deca\gram} &  & \SI{1}{\deci\gram} & \SI{1}{\centi\gram} & \SI{1}{\milli\gram}\\
= & = & = & \textbf{\SI{1}{\gram}} & = & = & =\\
\SI{1000}{\gram} & \SI{100}{\gram} & \SI{10}{\gram} & & \SI{0.1}{\gram} & \SI{0.01}{\gram} & \SI{0.001}{\gram}
\end{tabular}%
\end{center}

\noindent\(\SI{1}{\kilo\gram}\) is de massa van \(\SI{1}{\litre}\) zuiver water bij \(\SI{4}{\celsius}\). Verdere eenheden:
\begin{itemize}
  \item \(\SI{1}{\tonne}=\SI{1000}{\kilo\gram}\)
  \item \(\SI{1}{\carat}=\SI{2}{\deci\gram}\): de massa van diamant wordt uitgedrukt in karaat.
\end{itemize}

\subsection{Herleiden}
De herleiding van massa's gebeurt op dezelfde wijze als bij de lengtematen.

\subsubsection*{Voorbeeld 1}
\(\SI{0.05}{\kilo\gram}=\ldots\,\si{\deci\gram}\)
\[
\si{\kilo\gram}\xrightarrow[1]{}\si{\hecto\gram}\xrightarrow[2]{}\si{\deca\gram}\xrightarrow[3]{}\si{\gram}\xrightarrow[4]{}\si{\deci\gram}
\]
4 rangen naar rechts: komma 2 rangen opschuiven, 2 nullen bijplaatsen. Dus \(\SI{0.05}{\kilo\gram}=\SI{5}{\deca\gram}=\SI{500}{\deci\gram}\).

\subsubsection*{Voorbeeld 2}
\(\SI{2120}{\centi\gram}=\ldots\,\si{\deca\gram}\)
\[
\si{\deca\gram}\xleftarrow[3]{}\si{\gram}\xleftarrow[2]{}\si{\deci\gram}\xleftarrow[1]{}\si{\centi\gram}
\]
3 rangen naar links: komma 3 plaatsen naar links. Dus \(\SI{2120}{\centi\gram}=\SI{2.12}{\deca\gram}\).

\subsection*{A.C.O.~3}
\begin{multicols}{2}
  \begin{enumerate}
  \item \(\SI{17.5}{\kilo\gram}=\ldots\,\si{\deca\gram}\)
  \item \(\SI{0.204}{\hecto\gram}=\ldots\,\si{\deci\gram}\)
  \item \(\SI{50}{\deca\gram}=\ldots\,\si{\centi\gram}\)
  \item \(\SI{3.2}{\tonne}=\ldots\,\si{\hecto\gram}\)
  \item \(\SI{12.5}{\deca\gram}=\ldots\,\si{\kilo\gram}\)
  \item \(\SI{0.02}{\centi\gram}=\ldots\,\si{\hecto\gram}\)
  \item \(\SI{32150}{\milli\gram}=\ldots\,\si{\gram}\)
  \item \(\SI{200.5}{\deci\gram}=\ldots\,\si{\deca\gram}\)
  \end{enumerate}
\end{multicols}

\(\tfrac{1}{2}\,\si{\kilo\gram}=\ ?\)\\
\begin{tabular}{@{}ll@{}}
  A & \(\SI{200}{\gram}\) \\
  B & \(\SI{50}{\gram}\) \\
  C & \(\SI{0.20}{\gram}\) \\
  D & \(\SI{500}{\gram}\) \\
\end{tabular}

\medskip
\noindent Bij soortgelijke multiplechoice-oefeningen omcirkel je het juiste antwoord. Dit soort vragen wordt dikwijls gesteld op examens.

\section{Vlaktematen -- Landmaten}
Vlaktematen of oppervlaktematen worden gebruikt om de oppervlakte van een figuur uit te drukken.

\subsection{Overzicht van de vlaktematen}
De hoofdeenheid van de vlaktematen is de vierkante meter (\si{\square\metre}). Dit is de oppervlakte van een vierkant met zijde \(\SI{1}{\metre}\). Onderstaande tabel geeft een overzicht van de vlaktematen. Elke vlaktemaat is steeds het honderdvoud van de volgende in de tabel.

\paragraph{Omzettingstabel VLAKTEMATEN}
\renewcommand{\arraystretch}{1.2}
\begin{center}
\setlength{\fboxsep}{4pt}\setlength{\fboxrule}{0.5pt}%
\begin{tabular}{*{7}{c}}
  \multicolumn{3}{c}{\textbf{Veelvouden}} & \textbf{Eenheid} & \multicolumn{3}{c}{\textbf{Onderdelen}}\\
  \midrule
\SI{1}{\square\kilo\metre} & \SI{1}{\square\hecto\metre} & \SI{1}{\square\deca\metre} &  & \SI{1}{\square\deci\metre} & \SI{1}{\square\centi\metre} & \SI{1}{\square\milli\metre}\\
= & = & = & \textbf{\SI{1}{\square\metre}} & = & = & =\\
\SI{1000000}{\square\metre} & \SI{10000}{\square\metre} & \SI{100}{\square\metre} & & \SI{0.01}{\square\metre} & \SI{0.0001}{\square\metre} & \SI{0.000001}{\square\metre}
\end{tabular}%
\end{center}

\subsection{Herleiden}
\begin{itemize}
  \item Bij omzettingen naar de eerstvolgende kleinere oppervlakte-eenheid, verschuift de komma 2 rangen naar rechts.
  \item Bij omzettingen naar de eerstvolgende grotere oppervlakte-eenheid, verschuift de komma 2 rangen naar links.
\end{itemize}

\subsubsection*{Voorbeeld 1} \(\SI{5}{\square\deca\metre}=\SI{500}{\square\metre}\) (2 rangen naar rechts of 2 nullen bijplaatsen).

\subsubsection*{Voorbeeld 2} \(\SI{36000}{\square\centi\metre}=\ldots\,\si{\square\metre}\)
\[
\si{\square\metre} \xleftarrow[2]{}\si{\square\deci\metre} \xleftarrow[1]{}\si{\square\centi\metre}
\]
2 plaatsen naar links \(=2\times2\) rangen naar links. Dus \(\SI{36000}{\square\centi\metre}=\SI{3.6}{\square\metre}\).

\subsubsection*{Voorbeeld 3} \(\SI{0.04}{\square\hecto\metre}=\ldots\,\si{\square\deci\metre}\)
\[
\si{\square\hecto\metre} \xrightarrow[1]{}\si{\square\deca\metre} \xrightarrow[2]{}\si{\square\metre} \xrightarrow[3]{}\si{\square\deci\metre}
\]
3 plaatsen naar rechts \(=3\times2\) rangen \(=6\) rangen naar rechts. Dus \(\SI{0.04}{\square\hecto\metre}=\SI{40000}{\square\deci\metre}\).

\subsection*{A.C.O.~4}
\begin{multicols}{2}
\begin{enumerate}
  \item \(\SI{3}{\square\centi\metre}=\ldots\,\si{\square\metre}\)
  \item \(\SI{826}{\square\deci\metre}=\ldots\,\si{\square\metre}\)
  \item \(\SI{140000}{\square\metre}=\ldots\,\si{\square\kilo\metre}\)
  \item \(\SI{26.15}{\square\deca\metre}=\ldots\,\si{\square\kilo\metre}\)
  \item \(\SI{5}{\square\deca\metre}=\ldots\,\si{\square\centi\metre}\)
  \item \(\SI{0.26}{\square\kilo\metre}=\ldots\,\si{\square\deci\metre}\)
  \item \(\SI{83}{\square\kilo\metre}=\ldots\,\si{\square\metre}\)
  \item \(\SI{0.003}{\square\hecto\metre}=\ldots\,\si{\square\deci\metre}\)
\end{enumerate}
\end{multicols}

Men legt een vloer van \(\SI{26}{\square\metre}\) met tegels van \(\SI{10}{\centi\metre}\times\SI{20}{\centi\metre}\). Hoeveel tegels zijn er nodig?\\
\begin{tabular}{@{}ll@{}}
A & \num{13000} \\
B & \num{5200} \\
C & \num{1300} \\
D & \num{130} \\
\end{tabular}

\subsection{Landmaten}
De oppervlakte van akkers, weiden of bossen kan ook uitgedrukt worden in landmaten. De hoofdeenheid van de landmaat is de \emph{are} (\si{\are}). De are is de oppervlakte van \(\SI{1}{\square\deca\metre}=\SI{100}{\square\metre}\).

\paragraph{Omzettingstabel LANDMATEN}
\renewcommand{\arraystretch}{1.2}
\begin{center}
\setlength{\fboxsep}{4pt}\setlength{\fboxrule}{0.5pt}%
\begin{tabular}{*{3}{c}}
  \textbf{Veelvouden} & \textbf{Eenheid} & \textbf{Onderdelen} \\
  \midrule
  \(\SI{1}{\hectare}\)                &                  & \(\SI{1}{\centi\are}\)                \\
  =                   &                  & =                   \\
  \(\SI{100}{\are}\)               & \textbf{\(\SI{1}{\are}\)}     & \(\SI{0.01}{\are}\)              \\
  =                   &                  & =                   \\
  \(\SI{1}{\square\hecto\metre}\)            & \(\si{\square\deca\metre}\)          & \(\si{\square\metre}\)
\end{tabular}
\end{center}

\subsection{Herleidingen vlaktematen -- landmaten}

\subsection*{Tussen landmaten onderling}

Zelfde principe als bij de vlaktematen.

\subsubsection*{Voorbeeld}
\(\SI{0.40}{\hectare}=\ldots\,\si{\centi\are}\).
\[
\si{\hectare} \xrightarrow[1]{} \si{\are} \xrightarrow[2]{} \si{\centi\are}
\]
2 plaatsen naar rechts \(=2\times2\) rangen \(=4\) rangen. Dus \(\SI{0.40}{\hectare}=\SI{4000}{\centi\are}\).

\subsection*{Van landmaat naar vlaktemaat of omgekeerd}
We herleiden de opgave steeds tot de overeenstemmende vlaktemaat.

\subsubsection*{Voorbeeld 1}
\(\SI{0.25}{\square\hecto\metre}=\ldots\,\si{\centi\are}\)\\
\(\si{\centi\are} = \si{\square\metre}\) (de overeenstemmende vlaktemaat). \(\SI{0.25}{\square\hecto\metre}=\SI{2500}{\square\metre}\) (zie vlaktematen). Dus \(\SI{0.25}{\square\hecto\metre}=\SI{2500}{\centi\are}\).

\subsubsection*{Voorbeeld 2}
\(\SI{148000}{\are}=\ldots\,\si{\square\kilo\metre}\)\\
\(\si{\are}=\si{\square\deca\metre}\). Dus \(\SI{148000}{\square\deca\metre}=\SI{14.8}{\square\kilo\metre}\); bijgevolg \(\SI{148000}{\are}=\SI{14.8}{\square\kilo\metre}\).

\subsection*{A.C.O.~5}
\begin{multicols}{2}
\begin{enumerate}
  \item \(\SI{5}{\are}=\ldots\,\si{\centi\are}\)
  \item \(\SI{120}{\centi\are}=\ldots\,\si{\hectare}\)
  \item \(\SI{0.07}{\are}=\ldots\,\si{\hectare}\)
  \item \(\SI{0.8}{\hectare}=\ldots\,\si{\are}\)
\end{enumerate}
\columnbreak
\begin{enumerate}[start=5]
  \item \(\SI{7}{\are}=\ldots\,\si{\square\metre}\)
  \item \(\SI{2.15}{\hectare}=\ldots\,\si{\square\metre}\)
  \item \(\SI{2810}{\square\metre}=\ldots\,\si{\are}\)
  \item \(\SI{1.13}{\square\kilo\metre}=\ldots\,\si{\hectare}\)
\end{enumerate}
\end{multicols}

Welke oppervlakte is niet gelijk aan de andere?\\
\begin{tabular}{@{}ll@{}}
A & \(\SI{12}{\are}\,\SI{3}{\centi\are}\) \\
B & \(\SI{1203}{\square\metre}\) \\
C & \(\SI{0.123}{\hectare}\) \\
D & \(\SI{12.03}{\are}\) \\
\end{tabular}

\section{Inhoudsmaten -- Ruimtematen}
Inhoudsmaten en ruimtematen worden gebruikt om aan te duiden welk deel van de ruimte door een lichaam wordt ingenomen. Een boek, een boterdoosje, een fles wijn, \dots{} zijn voorbeelden van lichamen. Voorbeelden van meetkundige lichamen zijn: kubus, balk, cilinder, \dots{} Deze worden in les~3 verder besproken.

\subsection{Overzicht van de inhoudsmaten}
De hoofdeenheid van de inhoudsmaten is de liter. De liter is de inhoud van een kubus met zijde \(\SI{1}{\deci\metre}\) (1 kubieke decimeter). Onderstaande tabel geeft een overzicht van de inhoudsmaten. Bij twee opeenvolgende inhoudsmaten is de ene tienmaal groter dan de andere.

\paragraph{Omzettingstabel INHOUDSMATEN}
\renewcommand{\arraystretch}{1.2}
\begin{center}
\setlength{\fboxsep}{4pt}\setlength{\fboxrule}{0.5pt}%
\begin{tabular}{*{7}{c}}
  \multicolumn{3}{c}{\textbf{Veelvouden}} & \textbf{Eenheid} & \multicolumn{3}{c}{\textbf{Onderdelen}}\\
  \midrule
\si{\kilo\litre} & \si{\hecto\litre} & \si{\deca\litre} &  & \si{\deci\litre} & \si{\centi\litre} & \si{\milli\litre}\\
= & = & = & \textbf{\SI{1}{\litre}} & = & = & =\\
\SI{1000}{\litre} & \SI{100}{\litre} & \SI{10}{\litre} & & \SI{0.1}{\litre} & \SI{0.01}{\litre} & \SI{0.001}{\litre}
\end{tabular}
\end{center}

De benaming \emph{kiloliter} wordt zelden gebruikt. In dit geval schrijft men \(\SI{10}{\hecto\litre}\).

\subsection{Herleiden}
Dit gebeurt op dezelfde wijze als bij de lengtematen.

\subsubsection*{Voorbeeld 1}
\(\SI{15.3}{\deci\litre}=\ldots\,\si{\hecto\litre}\)\\
\[
\si{\hecto\litre} \xleftarrow[3]{}\si{\deca\litre} \xleftarrow[2]{}\si{\litre} \xleftarrow[1]{}\si{\deci\litre}
\]
3 rangen naar links: \(\SI{15.3}{\deci\litre}=\SI{0.0153}{\hecto\litre}\).

\subsubsection*{Voorbeeld 2}
\(\SI{56.14}{\deca\litre}=\ldots\,\si{\centi\litre}\)\\
\[
\si{\deca\litre} \xrightarrow[1]{}\si{\litre} \xrightarrow[2]{}\si{\deci\litre} \xrightarrow[3]{}\si{\centi\litre}
\]
3 rangen naar rechts: \(\SI{56.14}{\deca\litre}=\SI{56140}{\centi\litre}\).

\subsection*{A.C.O.~6}
\begin{multicols}{2}
\begin{enumerate}
  \item \(\SI{7}{\litre}=\ldots\,\si{\centi\litre}\)
  \item \(\SI{850}{\deci\litre}=\ldots\,\si{\hecto\litre}\)
\end{enumerate}
\columnbreak
\begin{enumerate}[start=3]
  \item \(\SI{0.06}{\centi\litre}=\ldots\,\si{\litre}\)
  \item \(\SI{0.35}{\litre}=\ldots\,\si{\centi\litre}\)
\end{enumerate}
\end{multicols}

\noindent\textbf{b)} \SI{1}{\deci\litre} is gelijk aan:\\
\begin{tabular}{@{}ll@{}}
A & een honderdste liter \\
B & een tiende liter \\
C & honderdmaal een liter \\
D & tienmaal een liter \\
\end{tabular}

\medskip
\noindent\textbf{c)} \(\SI{5}{\litre}+\SI{32}{\deci\litre}+\SI{310}{\milli\litre}=\ ?\)\\
\begin{tabular}{@{}ll@{}}
A & \(\SI{5.630}{\litre}\) \\
B & \(\SI{5.342}{\litre}\) \\
C & \(\SI{11.30}{\litre}\) \\
D & \(\SI{8.510}{\litre}\) \\
\end{tabular}

\subsection{De ruimtematen}
De hoofdeenheid van de ruimtematen is de kubieke meter (\si{\cubic\metre}). Dit is de ruimte die ingenomen wordt door een kubus met een ribbe van \(\SI{1}{\metre}\). Van twee opeenvolgende ruimtematen is de ene altijd duizendmaal groter dan de andere.

\paragraph{Omzettingstabel RUIMTEMATEN}
\renewcommand{\arraystretch}{1.2}
\begin{center}
\setlength{\fboxsep}{4pt}\setlength{\fboxrule}{0.5pt}%
\begin{tabular}{*{5}{c}}
  \textbf{Eenheid} & \multicolumn{3}{c}{\textbf{Onderdelen}}\\
  \midrule
& & \SI{1}{\cubic\deci\metre} & \SI{1}{\cubic\centi\metre} & \SI{1}{\cubic\milli\metre}\\
& \textbf{\SI{1}{\cubic\metre}} & = & = & =\\
& & \(\SI{0.001}{\cubic\metre}\) & \(\SI{1e-6}{\cubic\metre}\) & \(\SI{1e-9}{\cubic\metre}\)
\end{tabular}
\end{center}

Veelvouden van \(\si{\cubic\metre}\) worden in de praktijk uiterst zelden gebruikt.

\subsection{Herleiden van ruimtematen}
Elke eenheid gaat telkens duizendmaal in de daaropvolgende grotere eenheid. We schuiven de komma dus telkens drie rangen op.

\subsubsection*{Voorbeeld 1}
\(\SI{550}{\cubic\milli\metre}=\ldots\,\si{\cubic\deci\metre}\).\\
\[
\si{\cubic\deci\metre} \xleftarrow[2]{}\si{\cubic\centi\metre} \xleftarrow[1]{}\si{\cubic\milli\metre}
\]
2 plaatsen naar links \(=2\times3\) rangen \(=6\) rangen. Dus \(\SI{550}{\cubic\milli\metre}=\SI{0.00055}{\cubic\deci\metre}\).

\subsubsection*{Voorbeeld 2}
\(\SI{2.3}{\cubic\metre}=\ldots\,\si{\cubic\deci\metre}\).\\
\[
\si{\cubic\metre} \xrightarrow[1]{}\si{\cubic\deci\metre}
\]
1 plaats naar rechts \(=3\) rangen. Dus \(\SI{2.3}{\cubic\metre}=\SI{2300}{\cubic\deci\metre}\).

\subsection{Verband tussen inhoudsmaten en ruimtematen}
\[\SI{1}{\litre}=\SI{1}{\cubic\deci\metre}.\]

\subsection*{A.C.O.~7}
\begin{multicols}{2}
\begin{enumerate}
  \item \(\SI{0.75}{\cubic\deci\metre}=\ldots\,\si{\cubic\centi\metre}\)
  \item \(\SI{0.5}{\cubic\deci\metre}=\ldots\,\si{\cubic\milli\metre}\)
  \item \(\SI{8000}{\cubic\milli\metre}=\ldots\,\si{\cubic\deci\metre}\)
  \item \(\num{1000}\times \SI{6}{\cubic\centi\metre}=6\,\ldots\)
  \item \(\SI{80}{\cubic\deci\metre}+\ldots\,\si{\cubic\metre}=\SI{100}{\cubic\deci\metre}\)
  \item \(\SI{147}{\milli\litre}=\ldots\,\si{\cubic\milli\metre}\)
  \item \(\SI{320}{\litre}=\ldots\,\si{\cubic\metre}\)
  \item \(\SI{6}{\cubic\deci\metre}=\ldots\,\si{\centi\litre}\)
  \item \(\SI{120}{\cubic\centi\metre}=\ldots\,\si{\litre}\)
  \item \(\SI{3.4}{\hecto\litre}=\ldots\,\si{\cubic\centi\metre}\)
\end{enumerate}
\end{multicols}

\(\SI{1}{\litre}\) zuiver water bij \(\SI{4}{\celsius}\) weegt \(\SI{1}{\kilo\gram}\). Herleid voor zuiver water bij \(\SI{4}{\celsius}\):
\[
\begin{array}{rcl}
\SI{160}{\tonne} & \to & \ldots\,\si{\cubic\metre} \\
\SI{12.5}{\kilo\gram} & \to & \ldots\,\si{\litre} \\
\SI{16500}{\gram} & \to & \ldots\,\si{\deci\litre} \\
\SI{0.006}{\kilo\gram} & \to & \ldots\,\si{\deci\litre}
\end{array}
\]

\section*{OPLOSSINGEN A.C.O.}
\paragraph{1.}
\begin{multicols}{2}
\begin{enumerate}
  \item \(\SI{27.5}{\deci\metre}=\SI{2750}{\milli\metre}\)
  \item \(\SI{5}{\hecto\metre}=\SI{5000}{\deci\metre}\)
  \item \(\SI{5.66}{\metre}=\SI{5660}{\milli\metre}\)
  \item \(\SI{0.02}{\metre}=\SI{2}{\centi\metre}\)
  \item \(\SI{126}{\deci\metre}=\SI{0.0126}{\kilo\metre}\)
  \item \(\SI{5.3}{\centi\metre}=\SI{0.053}{\metre}\)
  \item \(\SI{3}{\centi\metre}=\SI{0.003}{\deca\metre}\)
  \item \(\SI{0.04}{\deca\metre}=\SI{0.004}{\hecto\metre}\)
\end{enumerate}
\end{multicols}
\paragraph{2.} \(\SI{999}{\metre};\quad \SI{6}{\centi\metre};\quad \SI{5}{\centi\metre};\quad \SI{330}{\centi\metre}.\)

\paragraph{3.}
\begin{multicols}{2}
\begin{enumerate}
  \item \(\SI{17.5}{\kilo\gram}=\SI{1750}{\deca\gram}\)
  \item \(\SI{0.204}{\hecto\gram}=\SI{204}{\deci\gram}\)
  \item \(\SI{50}{\deca\gram}=\SI{50000}{\centi\gram}\)
  \item \(\SI{3.2}{\tonne}=\SI{32000}{\hecto\gram}\)
  \item \(\SI{12.5}{\deca\gram}=\SI{0.125}{\kilo\gram}\)
  \item \(\SI{0.02}{\centi\gram}=\SI{0.000002}{\hecto\gram}\)
  \item \(\SI{32150}{\milli\gram}=\SI{32.15}{\gram}\)
  \item \(\SI{200.5}{\deci\gram}=\SI{2.005}{\deca\gram}\)
\end{enumerate}
\end{multicols}
\(\tfrac{1}{2}\,\si{\kilo\gram}\), Antwoord D.

\paragraph{4.}
\begin{multicols}{2}
\begin{enumerate}
  \item \(\SI{3}{\square\centi\metre}=\SI{0.0003}{\square\metre}\)
  \item \(\SI{826}{\square\deci\metre}=\SI{8.26}{\square\metre}\)
  \item \(\SI{140000}{\square\metre}=\SI{0.14}{\square\kilo\metre}\)
  \item \(\SI{26.15}{\square\deca\metre}=\SI{0.002615}{\square\kilo\metre}\)
  \item \(\SI{5}{\square\deca\metre}=\SI{5000000}{\square\centi\metre}\)
  \item \(\SI{0.26}{\square\kilo\metre}=\SI{26000000}{\square\deci\metre}\)
  \item \(\SI{83}{\square\kilo\metre}=\SI{83000000}{\square\metre}\)
  \item \(\SI{0.003}{\square\hecto\metre}=\SI{3000}{\square\deci\metre}\)
\end{enumerate}
\end{multicols}
Antwoord C. (Oppervlakte tegel \(=\SI{200}{\square\centi\metre}\), vloer \(=\SI{26}{\square\metre}=\SI{260000}{\square\centi\metre}\), aantal \(=\num{260000}:\num{200}=\num{1300}\).)

\paragraph{5.}
\begin{multicols}{2}
\begin{enumerate}
  \item \(\SI{5}{\are}=\SI{500}{\centi\are}\)
  \item \(\SI{120}{\centi\are}=\SI{0.012}{\hectare}\)
  \item \(\SI{0.07}{\are}=\SI{0.0007}{\hectare}\)
  \item \(\SI{0.8}{\hectare}=\SI{80}{\are}\)
  \item \(\SI{7}{\are}=\SI{700}{\square\metre}\)
  \item \(\SI{2.15}{\hectare}=\SI{21500}{\square\metre}\)
  \item \(\SI{2810}{\square\metre}=\SI{28.1}{\are}\)
  \item \(\SI{1.13}{\square\kilo\metre}=\SI{113}{\hectare}\)
\end{enumerate}
\end{multicols}
Antwoord C.

\paragraph{6.}
\begin{enumerate}
  \item \(\SI{7}{\litre}=\SI{700}{\centi\litre}\)
  \item \(\SI{850}{\deci\litre}=\SI{0.85}{\hecto\litre}\)
  \item \(\SI{0.06}{\centi\litre}=\SI{0.0006}{\litre}\)
  \item \(\SI{0.35}{\litre}=\SI{35}{\centi\litre}\)
\end{enumerate}
Antwoord B.\quad \; Antwoord D.

\paragraph{7.}
\begin{multicols}{2}
\begin{enumerate}
  \item \(\SI{0.75}{\cubic\deci\metre}=\SI{750}{\cubic\centi\metre}\)
  \item \(\SI{0.5}{\cubic\deci\metre}=\SI{500000}{\cubic\milli\metre}\)
  \item \(\SI{8000}{\cubic\milli\metre}=\SI{0.008}{\cubic\deci\metre}\)
  \item \(\num{1000}\times \SI{6}{\cubic\centi\metre}=\SI{6}{\cubic\deci\metre}\)
  \item \(\SI{80}{\cubic\deci\metre}+\ldots\,\si{\cubic\metre}=\SI{100}{\cubic\deci\metre}\)
  \item \(\SI{147}{\milli\litre}=\SI{147}{\cubic\centi\metre}\)
  \item \(\SI{320}{\litre}=\SI{0.32}{\cubic\metre}\)
  \item \(\SI{6}{\cubic\deci\metre}=\SI{600}{\centi\litre}\)
  \item \(\SI{120}{\cubic\centi\metre}=\SI{0.12}{\litre}\)
  \item \(\SI{3.4}{\hecto\litre}=\SI{340000}{\cubic\centi\metre}\)
\end{enumerate}
\end{multicols}
\(\SI{160}{\tonne}\to\SI{160}{\cubic\metre}\), \; \(\SI{12.5}{\kilo\gram}\to\SI{12.5}{\litre}\), \; \(\SI{16500}{\gram}\to\SI{165}{\deci\litre}\), \; \(\SI{0.006}{\kilo\gram}\to\SI{0.06}{\deci\litre}\).

\bigskip
\section*{Samenvatting}
\textbf{Herleiden}
\begin{itemize}
  \item Bij omzetting naar een kleinere eenheid verschuift de komma naar rechts.
  \item Bij omzetting naar een grotere eenheid verschuift de komma naar links.
  \item Bij lengtematen, inhoudsmaten en massa's verschuift de komma 1 rang per opeenvolgende eenheid.
  \item Bij landmaten en oppervlaktematen verschuift de komma telkens 2 rangen.
  \item Bij ruimtematen verschuift de komma telkens 3 rangen.
\end{itemize}

\end{document}

\documentclass[12pt]{article}

\textwidth 16cm \textheight 23cm \evensidemargin 0cm
\oddsidemargin 0cm \topmargin -2cm
\parindent 0pt
\parskip \medskipamount

\usepackage[utf8]{inputenc}
\usepackage[dutch]{babel}
\usepackage{amssymb}
\usepackage{amsmath}
\usepackage{amsthm}
\usepackage{hyperref}
\usepackage{enumerate}
\usepackage{subfig}
\usepackage{wrapfig}
\usepackage{minibox}
\usepackage{ifthen}
\usepackage{dot2texi}
\usepackage{multicol}
\usepackage{graphicx}
\usepackage{cancel}
%\usepackage{fix-cm}
\usepackage{setspace}
\usepackage{mdframed}
\usepackage{mathtools}
%\usepackage{lipsum}

\usepackage{exsol}
\renewcommand{\exercisename}{}
\renewcommand{\solutionname}{}

\usepackage{fancyhdr}
\pagestyle{fancy}

\usepackage{color}
\newcommand{\todo}[1]{\textcolor{red}{\##1\#}}
\newcommand{\question}[1]{\textcolor{blue}{\##1\#}}

\newcommand{\vraag}[2]{\begin{itemize}\item {\it #1} \vspace*{#2}\end{itemize}}

\newcommand{\degree}{\ensuremath{^\circ}}
\def\LRA{\Leftrightarrow\mkern40mu}

\newcommand\ggd{\qopname\relax o{\mathrm{ggd}}}
\newcommand\kgv{\qopname\relax o{\mathrm{kgv}}}

\newcounter{menucount}\newcounter{curitem}% Counters
\newcommand{\menuitem}{\texttt}% Menu item formatting
\newcommand{\menusep}{\ensuremath{\rightarrow}}% Menu separator
\newcommand{\menuend}{\relax}% Menu end
\newcommand{\menulist}[1]{% \menulist{<menu list>}
  \setcounter{menucount}{0}\setcounter{curitem}{0}% Reset menucount & curitem
  \renewcommand*{\do}[1]{\stepcounter{menucount}}%
  \menulistparser{#1}% Count menu items
  \renewcommand*{\do}[1]{\menuitem{##1}\stepcounter{curitem}\ifnumless{\value{curitem}}{\value{menucount}}{\menusep}{\menuend}}%
  \menulistparser{#1}% Process list
}
\DeclareListParser{\menulistparser}{:}% List separator is ':'

%\graphicspath{{../figuren/}}

\newcommand{\dotrule}[1]{%
   \parbox[t]{#1}{\vspace*{3pt}\dotfill}}
   
\newcommand{\dotFill}{\vspace*{6pt}\dotfill}

\newcommand{\dotlines}[1]{   
\foreach \n in {1,...,#1}{

\vspace*{0.1cm}
\dotfill
\vspace*{0.1cm}
}}

\newcommand{\ruitjes}[1]{

\hskip-2.6cm
\begin{tikzpicture}[scale=1.075,x=1.0cm,y=1.0cm]
\draw [help lines, solid, gray, very thin, step=0.5cm] (0,-#1+0.1cm) grid (21.6,-0.1);
\end{tikzpicture}
\vspace*{-1cm}
}

\newcommand{\ruitjesxy}[2]{
\begin{tikzpicture}[scale=1.01,line cap=round,line join=round,>=triangle 45,x=1.0cm,y=1.0cm]
\draw [color=cqcqcq,dash pattern=on 1pt off 1pt, xstep=0.4.5cm, ystep=0.4.5cm] (0,-#2) grid (#1,0);
\end{tikzpicture}
}

\newcommand{\zrmbox}{\framebox{\phantom{EXE}}\phantom{X}}
\newcommand{\zrm}[1]{\framebox{#1}}

% arule* answerrules
\def\arulefill{\xrfill[-0.5ex]{0.1pt}[lightgray]}
\newcommand{\arules}[1]{
\color{lightgray}
\vspace*{0.10cm}
\foreach \n in {1,...,#1}{
  \vspace*{0.70cm}
  \hrule height 0.1pt\hfill
}\color{black}}
\newcommand{\arule}[1]{
\color{lightgray}{\raisebox{-0.1cm}{\rule[-0.05cm]{#1}{0.1pt}}}\color{black}
}

% environment oefening:
% houdt een teller bij die de oefeningen nummert
% probeert ook de oefening op één pagina te houden
\newcounter{noefening}
\setcounter{noefening}{0}
\newenvironment{oefening}
{
  \stepcounter{noefening}
  \begin{minipage}{\textwidth}
  \vspace*{8pt}{\large\bf Oefening \arabic{noefening}}
}{%
  \end{minipage}
}

% environment voorbeeld:
% houdt een teller bij die de voorbeelden nummert
% nummering herbegint bij elke subsectie
% probeert het voorbeeld op één pagina te houden
\newcounter{nvb}[subsection]
%\@addtoreset{nvb}{subsubsection}
\setcounter{nvb}{0}
\newenvironment{voorbeeld}
{
  \stepcounter{nvb}
  \begin{minipage}{\textwidth}
  \vspace*{4pt}
  \textit{Voorbeeld \arabic{nvb}}\\[5pt]
}{%
  \end{minipage}
}

% environment voorbeeld*:
% probeert het voorbeeld op één pagina te houden
\newenvironment{voorbeeld*}
{
  \begin{minipage}{\textwidth}
  \vspace*{4pt}
  \textit{Voorbeeld}
}{%
  \end{minipage}
}

\newenvironment{onthoud}
{
\begin{mdframed}[nobreak=true,frametitle={Te onthouden}]
}{%
\end{mdframed}
}

\newcommand{\vglproef}[2]{LL= #1\;${=\joinrel=}$\; RL= #2}

\newcommand{\getallenas}[3][1]{
\definecolor{cqcqcq}{rgb}{0.65,0.65,0.65}
\begin{tikzpicture}[scale=#1,line cap=round,line join=round,>=triangle 45,x=1.0cm,y=1.0cm]
\draw [color=cqcqcq,dash pattern=on 1pt off 1pt, xstep=1.0cm,ystep=1.0cm] (#2,-0.2) grid (#3,0.2);
\draw[->,color=black] (#2,0) -- (#3,0);
\draw[shift={(0,0)},color=black] (0pt,2pt) -- (0pt,-2pt) node[below] {\footnotesize $0$};
\draw[shift={(1,0)},color=black] (0pt,2pt) -- (0pt,-2pt) node[below] {\footnotesize $1$};
\draw[color=black] (#3.25,0.07) node [anchor=south west] {$\mathbb{R}$};
\end{tikzpicture}
}

\newcommand{\opdracht}{{\bf Opdracht }}

% geef tabular iets meer ruimte
\setlength{\tabcolsep}{15pt}
\renewcommand{\arraystretch}{1.5}

% geef align iets meer ruimte:
\addtolength{\jot}{0.5em}

\newtheorem{definition}{Definitie}
\newtheorem{eigenschap}{Eigenschap}

\newcommand{\visgraad}[1]{\begin{tabular}{p{0.5cm}|p{#1}}&\\\hline\\\end{tabular}}

\newcommand{\assenstelsel}[5][1]{
\definecolor{cqcqcq}{rgb}{0.65,0.65,0.65}
\begin{tikzpicture}[scale=#1,line cap=round,line join=round,>=triangle 45,x=1.0cm,y=1.0cm]
\draw [color=cqcqcq,dash pattern=on 1pt off 1pt, xstep=1.0cm,ystep=1.0cm] (#2,#4) grid (#3,#5);
\draw[->,color=black] (#2,0) -- (#3,0);
\draw[shift={(1,0)},color=black] (0pt,2pt) -- (0pt,-2pt) node[below] {\footnotesize $1$};
\draw[color=black] (#3.25,0.07) node [anchor=south west] { x};
\draw[->,color=black] (0,#4) -- (0,#5);
\draw[shift={(0,1)},color=black] (2pt,0pt) -- (-2pt,0pt) node[left] {\footnotesize $1$};
\draw[color=black] (0.09,#5.25) node [anchor=west] { y};
\draw[color=black] (0pt,-10pt) node[right] {\footnotesize $0$};
\end{tikzpicture}
}

\newcommand{\ConfigureExSol}{
\renewcommand{\exercisename}{A.C.O.}
\renewcommand{\solutionname}{A.C.O.}
\newcounter{exercise2}[section]
\setcounter{exercise2}{0}
\renewcommand{\theexercise}{%
    \arabic{exercise2}%
}
\renewenvironment{exsol@exercise}[0]
{%
\refstepcounter{exercise2}
  \begin{minipage}[t]{\textwidth}%
    \ifthenelse{\boolean{exsol@exerciseaslist}}
               {\begin{list}%
                   {%
                   }%
                   {%
                     \setlength{\topsep}{0pt}%
                     \setlength{\leftmargin}{1em}%
                     \setlength{\rightmargin}{1em}%
                     \setlength{\listparindent}{0em}%
                     \setlength{\itemindent}{0em}%
                     \setlength{\parsep}{\parskip}}%
                 \item[\hspace*{\leftmargin}\textit{\exercisename{}
                                                    \theexercise:}]
               }%
               {
                 \textbf{\exercisename{} \theexercise:}~
               }
}
{%
  \ifthenelse{\boolean{exsol@exerciseaslist}}
             {\end{list}}{}
  \end{minipage}
  \vspace{1ex}\par
}
}

\onehalfspacing
%singlespacing
%doublespacing



\usepackage{tikz}
\usetikzlibrary{calc,arrows}
\newcommand{\tikzmark}[1]{\tikz[overlay,remember picture] \node (#1) {};}

%%%%%%%%%%%%%%%%%%%%%%%%%%%%%%%%%%%%%%%%%%
% Volgende wijzigingen werden aangebracht tov de originele bis-basis
% * gebruik linkerlid en rechterlid ipv eerste lid en tweede lid
% * extra voorbeelden bij oplossen van een vergelijking uit het hoofd
% * gebruik van equivalent met (<=>) waar gepast

%%%%%%%%%%%%%%%%%%%%%%%%%%%%%%%%%%%%%%%%%%
% * volgende opmerkingen heb ik nog
% * er wordt niet gesproken over referentieverzameling
% * de oplossingenverzameling wordt niet genoteerd

\lhead{}
\rhead{BIS-Basis -- Les 16}

\begin{document}

\ConfigureExSol

\setcounter{section}{15}
\section{Vergelijkingen}


\subsection{Begrippen}

Deze les sluit nauw aan bij les 6, gelijkheden en ongelijkheden.
$$5 + 4 = 9 \mbox{ is een gelijkheid.}$$
Indien een term uit een gelijkheid onbekend is, noemen we deze gelijkheid een {\bf vergelijking}.
$$5 + ? = 9$$
Het onbekende getal wordt voorgesteld door een letter, meestal $x$.
$$5 + x = 9$$
In deze vergelijking is $5 + x$ het {\bf linkerlid} en $9$ het {\bf rechterlid}.
$x$ is de {\bf onbekende}.
Vervang je $x$ door $4$, dan bekom je een gelijkheid. We noemen $4$ een {\bf oplossing} van deze vergelijking.
Een vergelijking oplossen betekent de waarde van de onbekende $x$ bepalen, zodat de vergelijking een gelijkheid wordt.

\subsection{Oplossen van een vergelijking uit het hoofd}

Sommige vergelijkingen kan je gemakkelijk uit het hoofd oplossen.

\begin{voorbeeld*}
\begin{center}
  \begin{tabular}{c|c|c}
  Vergelijking & Oplossing & Reden\\
  \hline
  $x+6 = 13$ & $x=7$ & $7+6=13$\\
  $30 - x = 16$ &	$x = 14$ & $30 - 14 = 16$\\
  $\dfrac{x}{5} = 20$ & $x = 100$	& $100 : 5 = 20$\\
  $3x = 27$	& $x = 9$	&	$3\cdot 9 = 27$\\
  $x^2=4$ & $x=-2$ of $x=2$ & $(-2)^2=4$ en $2^2=4$\\
  $\sqrt{x}=3$ & $x=9$ & $\sqrt{9}=3$\\
  $\dfrac{5}{x}=1$ & $x=5$ & $\dfrac{5}{5}=1$
  \end{tabular}
\end{center}
\end{voorbeeld*}

Let op: In een vergelijking wordt het maalteken voorgesteld door een ‘$\cdot$’ of, indien mogelijk, weggelaten. Dit doet men om verwarring met de onbekende $x$ te vermijden.

\subsection{Basisregels voor het oplossen van vergelijkingen}

Moeilijkere vergelijkingen, bv. $3x - 7 = 5 \cdot (2x + 3)$ kan je niet uit het hoofd oplossen. Hiervoor bestaat een oplossingsmethode.
Deze methode wordt stap voor stap uitgewerkt, beginnend vanuit eenvoudige voorbeelden.

\subsubsection{Term overbrengen}

\begin{voorbeeld}
\begin{align*}
x + 9 &= - 5
\intertext{Een vergelijking oplossen is de waarde van $x$ bepalen.
We zorgen ervoor dat $x$ in het linkerlid staat en alle andere termen in het rechterlid.}
x + 9 - 9 &= - 5 - 9 \qquad\mbox{(links en rechts dezelfde bewerking)}
\intertext{We trekken van beide leden $9$ af. Zo bekomen we opnieuw een gelijkheid.}
x &= - 5 - 9 	\qquad(+ 9 - 9 = 0)\\
x &= - 14
\end{align*}
Om te controleren of de bekomen oplossing klopt, vervang je, in de oorspronkelijke vergelijking, $x$ door het bekomen resultaat en controleer je of je een gelijkheid bekomt.

{\em Controle: }\\
$$
\begin{rcases*}
\mbox{LL } = x + 9 = - 14 + 9 = - 5\;\\
\mbox{RL } = - 5
\end{rcases*} \Rightarrow \mbox{LL ${=\joinrel=}$ RL, dus OK}
$$

Om een {\bf term} uit een lid te verwijderen, vermeerder je beide leden met het tegengestelde van die term.

{\em Praktische werkwijze: }\\
\begin{alignat*}{2}
     &&x + \underline{9}_{\tikzmark{a}}	&= - 5\tikzmark{b}\\
\LRA &&     x	&= - 5 - 9\\
\LRA &&     x	&= - 14
\begin{tikzpicture}[overlay,remember picture,out=315,in=225,distance=0.5cm]
\draw[-stealth,shorten >=4pt,shorten <=-2pt] (a.center) to (b.center);
\end{tikzpicture}
\end{alignat*}

{\em Opmerking:}\\
De drie bovenstaande vergelijkingen zijn equivalent. Dit wil zeggen dat ze alle drie dezelfde oplossingen hebben. We duiden dit aan met een '$\Leftrightarrow$'.
\end{voorbeeld}

\begin{voorbeeld}
\begin{align*}
x - 17 &= 6
\intertext{Breng de term zonder $x$ naar het rechterlid. Doe dit door links en rechts 17 op te tellen.}
x - 17 + 17 &= 6 + 17\\
x &= 23
\end{align*}
{\em Controle: } \vglproef{23-17=6}{6}

{\em Praktische werkwijze:}\\
\begin{alignat*}{2}
     &&x \underline{- 17}_{\tikzmark{a}}	&= 6\tikzmark{b}\\
\LRA &&x	    &= 6 + 17\\
\LRA &&x	    &= 23
\begin{tikzpicture}[overlay,remember picture,out=315,in=225,distance=0.5cm]
\draw[-stealth,shorten >=4pt,shorten <=-2pt] (a.center) to (b.center);
\end{tikzpicture}
\end{alignat*}

\end{voorbeeld}

\subsubsection{Factor overbrengen}


\begin{voorbeeld}
\begin{alignat*}{2}
     && 3x &= - 24\\
\intertext{De onbekende $x$ wordt links vermenigvuldigd met 3, we kunnen dit vermijden door beide leden te delen door 3.}
\LRA && 3x\cdot \dfrac{1}{3} &= -24\cdot \dfrac{1}{3}\\
\LRA && x &= -8
\end{alignat*}

{\em Controle: } \vglproef{$3\cdot(-8)=-24$}{$-24$}

{\em Praktische werkwijze:}\\

\begin{alignat*}{2}
     && \tikzmark{a}3\cdot x &= -24\tikzmark{b}\\
\LRA &&   x &= -24 : 3\\
\LRA &&   x &= -8
\begin{tikzpicture}[overlay,remember picture,out=315,in=225,distance=0.5cm]
\draw[-stealth,shorten >=4pt,shorten <=4pt] (a.center) to (b.center);
\end{tikzpicture}
\end{alignat*}

Als een factor van lid verandert, deel je door die factor.

{\em Controle: } \vglproef{$3\cdot -8=-24$}{$-24$}

\end{voorbeeld}


\begin{voorbeeld}
\begin{alignat*}{2}
     && \underline{-5}_{\tikzmark{a}}\cdot x &= \tikzmark{b}\dfrac{2}{7}\\
\LRA &&   x &= \dfrac{2}{7}\cdot\dfrac{-1}{5}\\
\LRA &&   x &= -\dfrac{2}{35}
\end{alignat*}

\begin{tikzpicture}[overlay,remember picture,out=315,in=225,distance=0.5cm]
\draw[-stealth,shorten >=4pt,shorten <=-2pt] (a.center) to (b.center);
\end{tikzpicture}

Als een factor van lid verandert kunnen we in plaats van te delen door die factor ook vermenigvuldigen met de omgekeerde van de factor.

{\em Controle: } \vglproef{$-5\cdot\dfrac{-2}{35}=\dfrac{2}{7}$}{$\dfrac{2}{7}$}
\end{voorbeeld}

\begin{voorbeeld}

\begin{minipage}{0.5\textwidth}
\begin{alignat*}{2}
    && x : (-4) &= 9\\
\LRA&& x &= 9 \cdot(-4)\\
\LRA&&x &= -36
\end{alignat*}
\end{minipage}
\begin{minipage}[t]{0.5\textwidth}
Delen door $-4$ is gelijk aan vermenigvuldigen met $-\frac{1}{4}$. Veranderen van lid is vermenigvuldigen met het omgekeerde. Het omgekeerde van $-\frac{1}{4}$ is $-4$. M.a.w. delen door $-4$ wordt bij overbrenging vermenigvuldigen met $-4$.
\end{minipage}
\end{voorbeeld}
{\em Controle: }\\\vglproef{x:(-4)=-36:(-4)=9}{9}

\subsubsection{Gemengde oefeningen}

Indien er in een oefening verschillende termen met $x$ voorkomen, stelt deze $x$ steeds hetzelfde getal voor.

Breng alle termen waarin een $x$ voorkomt naar het linkerlid, alle termen zonder $x$ naar het rechterlid.

\newcommand{\underwrite}[2]{
\underset{#2}{\underline{#1}}
}

\begin{voorbeeld}
\begin{alignat*}{2}
     && x-\underwrite{5}{\rightarrow}  &= 7-\underwrite{3x}{\leftarrow}\\
\LRA && x + 3x &= 7 + 5 \\
\LRA && 4x &= 12 \qquad \text{Maak in beide leden de som van de termen.}\\
\LRA && x&=\dfrac{12}{4} \qquad\text{Deel beide leden door 4.}\\
\LRA && x&=3 \\
\end{alignat*}
{\em Controle: }\\\vglproef{$x-5=3-5=-2$}{$7-3x=7-3\cdot3=7-9=-2$}
\end{voorbeeld}

\begin{voorbeeld}
\begin{alignat*}{2}
     && 2x\underwrite{-5}{\rightarrow}  &= \underwrite{5x}{\leftarrow}+7\\
\LRA && 2x-5x &= 7+5 \qquad\text{Alle onbekenden naar links, alle bekenden naar rechts.} \\
\LRA && -3x   &= 12 \qquad \text{Bereken de som van de termen.}\\
\LRA && x     &=\dfrac{12}{-3} \qquad\text{De factor $-3$ verplaatsen, let op: het teken wijzigt niet.}\\
\LRA && x     &= -4 \\
\end{alignat*}
{\em Controle: }\\\vglproef{$2(-4)-5=-13$}{$5(-4)+7=-13$}
\end{voorbeeld}

Let op bij de controle: vul de gevonden oplossing steeds in de oorspronkelijke opgave in voordat je enige bewerking uitvoert.

\begin{exercise}
Los volgende vergelijkingen op:
\begin{enumerate}[(a)]
  \item $x+3=-10$
  \item $-3x=-12$
  \item $- 2x + 12 = - 20$
  \item $9x + 7 = 3 - 11x$
  \item $0,9x = 6 + 0,3x$
\end{enumerate}
\end{exercise}

\begin{solution}
\vspace{-2\topsep}
\begin{enumerate}[(a)]
\item $\begin{aligned}[t]
     && x+3 &= -10\\
\LRA &&   x &= -10 -3\\
\LRA &&   x &= -13
\end{aligned}$
\item $\begin{aligned}[t]
     && -3x &= -12\\
\LRA &&   x &= \dfrac{-12}{-3}\\
\LRA &&   x &= 4
\end{aligned}$
\item $\begin{aligned}[t]
     && - 2x + 12	&= - 20\\
\LRA &&      - 2x &= - 20 - 12\\
\LRA &&      - 2x &= - 32\\
\LRA &&        x	&= \dfrac{-32}{-2}\\
\LRA &&        x  &= 16
\end{aligned}$
\item $\begin{aligned}[t]
     && 9x + 7 &= 3 - 11x\\
\LRA && 9x + 11x	&= 3 - 7\\
\LRA && 20x	&= - 4\\
\LRA && x &= \dfrac{-4}{20}\\
\LRA && x &= -\dfrac{1}{5}\\
\end{aligned}$
\item $\begin{aligned}[t]
     && 0,9x	&= 6 + 0,3x \\
\LRA && 0,9x - 0,3x	&= 6\\
\LRA && 0,6x	&= 6\\
\LRA && x	&= \dfrac{6}{0.6} \\
\LRA && x &= 10\\
\end{aligned}$
\end{enumerate}
\end{solution}

\subsection{Vergelijkingen met haakjes}
Indien er in een vergelijking haakjes voorkomen, werken we die eerst weg.

\begin{voorbeeld}
\begin{alignat*}{2}
     && - 2 (5 - x)	&= 7 (x + 5)\\
\intertext{een getal maal een som: verdrijf de haakjes door toepassing van de distributieve eigenschap}
\LRA && \underwrite{-10}{\rightarrow} + 2x &= \underwrite{7x}{\leftarrow} + 35 \\
\LRA && 2x - 7x	&= 35 + 10\\
\LRA && - 5x	&= 45\\
\LRA && x	&= 45 : (- 5) \\
\LRA && x &= -9
\end{alignat*}
{\em Controle: }\\\vglproef{$-2(5-x)=-2(5-(-9))=-28$}{$7(x+5)=7(-9+5)=-28$}
\end{voorbeeld}

\begin{voorbeeld}
\begin{alignat*}{2}
     && 8 - (4x + 5) &= 2x + 6 + (7 - x)\\
\intertext{+ of - voor de haakjes : verdrijf de haakjes door toepassing van de haakjesregel}
\LRA && \underwrite{8}{\rightarrow}-4x\underwrite{-5}{\rightarrow} &= \underwrite{2x}{\leftarrow}+6+7\underwrite{-x}{\leftarrow} \\
\LRA && 	- 4x - 2x + x	&= 6 + 7 - 8 + 5 \\
\LRA &&  - 5x	&= 10\\
\LRA &&  x	&= 10 : (- 5)\\
\LRA &&  x	&= - 2\\
\end{alignat*}
{\em Controle: }\\\vglproef{$8 - (4 (- 2) + 5 )= 11$}{$2 (- 2) + 6 + (7 - (- 2))=11$}
\end{voorbeeld}

\begin{exercise}
Los volgende vergelijkingen op en maak de proef:
\begin{enumerate}[(a)]
  \item $4 (- x + 5) = 2 (3x - 7)$
  \item $7 - (2x - 5) = - 2 (5x + 3)$
  \item $(7 - x) - 2 (x + 1) = 29 + x$
  \item $40 - 3 (x + 3) = 60 - (5x + 15)$
\end{enumerate}
\end{exercise}

\begin{solution}
\vspace{-2\topsep}
\begin{enumerate}[(a)]
\item $\begin{aligned}[t]
     && 4(- x + 5)	&= 2(3x - 7)\\
\LRA &&   - 4x + 20	&= 6x - 14\\
\LRA &&   - 4x - 6x	&= -14 - 20\\
\LRA &&   - 10x	&= -34\\
\LRA &&   x	&= 3.4\\
\end{aligned}$
\item $\begin{aligned}[t]
     && 7 - (2x - 5)	&= - 2(5x + 3)\\
\LRA && 7 - 2x + 5	&= - 10x - 6\\
\LRA && - 2x + 10x	&= - 6 - 7 - 5\\
\LRA && 8x	&= - 18\\
\LRA && x	&= \dfrac{-18}{8}\\
\LRA && x	&= -\dfrac{9}{4}\\
\end{aligned}$
\item $\begin{aligned}[t]
     && (7 - x) - 2(x + 1)	&= 29 + x\\
\LRA && 7 - x - 2x - 2	&= 29 + x\\
\LRA && - x - 2x - x	&= 29 - 7 + 2\\
\LRA && - 4x	&= 24\\
\LRA && x	&= - 6\\
\end{aligned}$
\item $\begin{aligned}[t]
     && 40 - 3(x + 3)	&= 60 - (5x + 15)\\
\LRA && 40 - 3x - 9	&= 60 - 5x - 15\\
\LRA && - 3x + 5x	&= 60 - 15 - 40 + 9\\
\LRA && 2x	&= 14\\
\LRA && x	&= 7\\
\end{aligned}$
\end{enumerate}
\end{solution}

\subsection{Vergelijkingen met breuken}

\begin{voorbeeld}
\begin{alignat*}{2}
     && \dfrac{4}{5}x-\dfrac{1}{2}&=\dfrac{2}{3}x+7\\
\intertext{7 herschrijf je als $\dfrac{7}{1}$, maak dan alle breuken gelijknamig}
\LRA && \dfrac{24}{30}x-\dfrac{15}{30}&=\dfrac{20}{30}x+\dfrac{210}{30}\\
\intertext{Vermenigvuldig beide leden met de gemeenschappelijke noemer 30.
	Als we beide leden van een gelijkheid vermenigvuldigen met eenzelfde getal, blijft de gelijkheid behouden.}
\LRA && 30\cdot\left(\dfrac{24}{30}x-\dfrac{15}{30}\right)&=30\cdot\left(\dfrac{20}{30}x+\dfrac{210}{30}\right)\\
\intertext{Werk de haakjes uit (distributieve eigenschap) en vereenvoudig iedere term.}
\LRA && \dfrac{30\cdot24}{30}x-\dfrac{30\cdot15}{30}&=\dfrac{30\cdot20}{30}x+\dfrac{30\cdot210}{30}\\
\LRA && 24x-15 &= 20x+210 \\
\intertext{Als we beide leden vermenigvuldigen met de noemer, valt de noemer weg.
	Vergelijk de laatste regel met de tweede regel uit de oefening.
	Werk nu verder uit op de gewone manier.}
\LRA && 24x-20x &= 210+15 \\
\LRA && 4x &= 225 \\
\LRA && x &= \dfrac{225}{4}
\end{alignat*}
{\em Controle: }\\\vglproef{$\dfrac{4}{5}\cdot\dfrac{225}{4}-\dfrac{1}{2}=\dfrac{89}{2}$}{$\dfrac{2}{3}\cdot\dfrac{225}{4}+7=\dfrac{89}{2}$}
\end{voorbeeld}

\begin{voorbeeld}
\begin{alignat*}{2}
     && 6-\dfrac{4}{3}x &= 11-\dfrac{7}{4}x \qquad\text{Noemer $= 12$}\\
\LRA && \dfrac{72}{12}-\dfrac{16}{12}x &= \dfrac{132}{12}-\dfrac{21}{12}x \qquad\text{Noemer verdrijven ($\times 12$)}\\
\LRA && \underwrite{72}{\rightarrow}-16x&=132\underwrite{-21x}{\leftarrow}\\
\LRA && - 16x + 21x	&= 132 - 72\\
\LRA && 5x	&= 60\\
\LRA && x	&= \dfrac{60}{5}\\
\LRA && x	&= 12
\end{alignat*}
{\em Controle: }\\\vglproef{$6-\dfrac{4}{3}\cdot 12 = -10$}{$11-\dfrac{7}{4}x=-10$}
\end{voorbeeld}

\begin{voorbeeld}
\begin{alignat*}{2}
     && 2x-5 &= \dfrac{x-2}{3} + \dfrac{2}{3} \qquad\text{Noemer $= 3$}\\
\LRA && \dfrac{6x}{3}-\dfrac{15}{3} &= \dfrac{x-2}{3} + \dfrac{2}{3} \qquad\text{Noemer verdrijven ($\times 3$)}\\
\LRA && 6x - 15	&= x - 2 + 2\\
\LRA && 6x - x	&= 15\\
\LRA && 5x	&= 15\\
\LRA && x	&= 3\\
\end{alignat*}
{\em Controle: }\\\vglproef{$2\cdot 3 - 5=1$}{$\dfrac{3-2}{3}+\dfrac{2}{3}=1$}
\end{voorbeeld}

\begin{onthoud}
Vergelijkingen waarin breuken voorkomen:
\begin{itemize}
  \item maak de breuken gelijknamig
  \item werk de noemer weg, door beide leden te vermenigvuldigen met de gemeenschappelijke noemer
  \item werk verder uit
\end{itemize}
\end{onthoud}

\begin{voorbeeld}
Een vergelijking waarbij een minteken voor de breukstreep staat, verdient bijzondere aandacht.
\begin{alignat*}{2}
     && 8-\dfrac{2x-1}{3} &= x \qquad\text{Maak gelijknamig: noemer $= 3$}\\
\LRA && \dfrac{24}{3}-\dfrac{2x-1}{3} &= \dfrac{3x}{3}\\
\intertext{Verdrijf de noemer. Let op het minteken voor de breukstreep. Dit minteken hoort bij alle termen uit de teller. Daarom zetten we de teller tussen haakjes.}
\LRA && 24 - (2x - 1)	&= 3x	\qquad\text{Verdrijf de haakjes.}\\
\LRA && 24 - 2x + 1	&= 3x\\
\LRA && - 2x - 3x	&= - 1 - 24 \qquad\text{Groepeer de termen.}\\
\LRA && - 5x	&= - 25\\
\LRA && x	&= \dfrac{25}{5}\\ 
\LRA && x	&= 5\\
\end{alignat*}
{\em Controle: }\\\vglproef{$8-\dfrac{2\cdot 5-1}{3}=5$}{$5$}
\end{voorbeeld}

\begin{voorbeeld}
\begin{alignat*}{2}
     && 10 - \dfrac{x-2}{5} &= \dfrac{8(15-x)}{3} \qquad\text{Verdrijf de haakjes d.m.v. de distr. eigenschap.}\\
\LRA && 10 - \dfrac{x-2}{5} &= \dfrac{120-8x}{3} \qquad\text{Maak de breuken gelijknamig.}\\
\LRA && \dfrac{150}{15} - \dfrac{3x-6}{15} &= \dfrac{600-40x}{15}	\qquad\text{Werk de noemer weg.}\\
\LRA && 150 - (3x -6) &= 600 - 40x\\
\LRA && 150 - 3x + 6	&= 600 - 40x\\
\LRA && - 3x + 40x	&= 600 - 150 - 6\\
\LRA && 37x	&= 444\\
\LRA && x	&= \dfrac{444}{37}\\
\LRA && x	&= 12
\end{alignat*}
{\em Controle: }\\\vglproef{$10-\dfrac{12-2}{5}=8$}{$\dfrac{8(15-12)}{3}=8$}
\end{voorbeeld}

\begin{exercise}
Los volgende vergelijkingen op:
\begin{enumerate}[(a)]
  \item $\dfrac{x+2}{6}=\dfrac{x-5}{4}$
  \item $\dfrac{4}{3}x-\dfrac{5}{2}=\dfrac{1}{5}x-\dfrac{1}{2}+\dfrac{1}{3}x$
  \item $\dfrac{x-4}{3}-\dfrac{x+3}{5}=\dfrac{x}{15}$
  \item $x-\dfrac{3x+3}{6}=\dfrac{3(x+1)}{4}$
\end{enumerate}
\end{exercise}

\begin{solution}
\vspace{-2\topsep}
\begin{enumerate}[(a)]
\item $\begin{aligned}[t]
     && \dfrac{x+2}{6}&=\dfrac{x-5}{4}\\
\LRA && \dfrac{2x+4}{12}&=\dfrac{3x-15}{12}\\
\LRA && 2x + 4	&= 3x - 15\\
\LRA && 2x - 3x	&= - 15 - 4\\
\LRA && - x	&= - 19\\
\LRA && x	&= 19\\
\end{aligned}$
\item $\begin{aligned}[t]
     && \dfrac{4}{3}x-\dfrac{5}{2}&=\dfrac{1}{5}x-\dfrac{1}{2}+\dfrac{1}{3}x\\
\LRA && \dfrac{40}{30}x-\dfrac{75}{30}&=\dfrac{6}{30}x-\dfrac{15}{30}+\dfrac{10}{30}x\\
\LRA && 40x - 75	&= 6x - 15 + 10x\\
\LRA && 40x - 6x - 10x	&= - 15 + 75\\
\LRA && 24x	&= 60\\
\LRA && x	&= \dfrac{60}{24}\\
\LRA && x	&= \dfrac{5}{2}\\
\end{aligned}$
\item $\begin{aligned}[t]
     && \dfrac{x-4}{3}-\dfrac{x+3}{5}&=\dfrac{x}{15}\\
\LRA && \dfrac{5x-20}{15}-\dfrac{3x+9}{15}&=\dfrac{x}{15}\\
\LRA && 5x - 20 - (3x + 9)	&= x\\
\LRA && 5x - 20 - 3x - 9	&= x\\
\LRA && 5x - 3x - x	&= 20 + 9\\
\LRA && x	&= 29\\
\end{aligned}$
\item $\begin{aligned}[t]
     && x-\dfrac{3x+3}{6}&=\dfrac{3(x+1)}{4}\\
\LRA && x-\dfrac{3x+3}{6}&=\dfrac{3x+3)}{4}\\
\LRA && \dfrac{12x}{12}-\dfrac{6x+6}{12}&=\dfrac{9x+9)}{12}\\
\LRA && 12x - 6x - 6	&= 9x + 9\\
\LRA && 12x - 6x - 9x	&= 9 + 6\\
\LRA && - 3x	&= 15\\
\LRA && x	&= - 5\\
\end{aligned}$
\end{enumerate}
\end{solution}

\subsection{Aantal oplossingen van een vergelijking}

Alle vergelijkingen die we tot nu toe oplosten hebben juist 1 oplossing. We noemen ze {\bf echte vergelijkingen}. Er bestaan ook andere vergelijkingen.

\begin{voorbeeld}
\begin{alignat*}{2}
     && 3x + 18 + 6x &= 9x + 12\\
\LRA && 3x + 6x - 9x &= 12 - 18\\
\LRA && 0x	&= - 6
\end{alignat*}
\end{voorbeeld}

Deze vergelijking heeft geen oplossing, want je kan voor $x$ geen enkel getal invullen opdat $0$ keer dit getal gelijk zou zijn aan $-6$. Dit is een {\bf valse vergelijking}.

\begin{voorbeeld}
\begin{alignat*}{2}
     && 8x - 6	&= 2(3x - 3) + 2x\\
\LRA && 8x - 6	&= 6x - 6 + 2x\\
\LRA && 8x - 6x - 2x	&= - 6 + 6\\
\LRA && 0x	&= 0
\end{alignat*}
\end{voorbeeld}

Voor $x$ kan je alle getallen invullen. Als je een willekeurig getal vermenigvuldigt met 0, bekom je altijd 0. 0 is het opslorpend element voor de vermenigvuldiging. Deze vergelijking heeft oneindig veel oplossingen. We noemen dit een {\bf identieke vergelijking}.

\begin{onthoud}
\begin{itemize}
  \item Een {\bf echte} vergelijking heeft juist één oplossing.
  \item Een {\bf valse} vergelijking heeft geen enkele oplossing.
  \item Een {\bf identieke} vergelijking heeft oneindig veel oplossingen.
\end{itemize}
\end{onthoud}

\begin{exercise}
Los volgende vergelijkingen op en vermeld of ze echt, vals of identiek zijn.
\begin{enumerate}[(a)]
  \item $3x + 7 = 5x - 3 - 2x$
  \item $\dfrac{1}{7}(5x+9) = x - 1$
  \item $\dfrac{1}{2}(x-30)=2(\dfrac{x}{3}-5)-\dfrac{1}{3}(\dfrac{x}{2}+15)$
\end{enumerate}
\end{exercise}

\begin{solution}
\vspace{-2\topsep}
\begin{enumerate}[(a)]
\item $\begin{aligned}[t]
     && 3x + 7 &= 5x - 3 - 2x\\
\LRA && 3x - 5x + 2x	&= - 3 - 7\\
\LRA && 0x	&= - 10 \\
\end{aligned}$\\
Valse vergelijking.
\item $\begin{aligned}[t]
     && \dfrac{1}{7}(5x+9) &= x - 1\\
\LRA && \dfrac{5}{7}x+\dfrac{9}{7} &= \dfrac{7}{7}x-\dfrac{7}{7}\\
\LRA && 5x + 9	&= 7x - 7\\
\LRA && 5x - 7x	&= - 7 - 9\\
\LRA && - 2x	&= - 16\\
\LRA && x	&= \dfrac{-16}{-2}\\
\LRA && x	= 8\\
\end{aligned}$\\
Echte vergelijking.
\item $\begin{aligned}[t]
     && \dfrac{1}{2}(x-30)&=2(\dfrac{x}{3}-5)-\dfrac{1}{3}(\dfrac{x}{2}+15)\\
\LRA && \dfrac{x}{2}-\dfrac{30}{2}&=\dfrac{2x}{3}-10-\dfrac{x}{6}+\dfrac{15}{3}\\
\LRA && \dfrac{3x}{6}-\dfrac{90}{3}&=\dfrac{4x}{6}-\dfrac{60}{6}-\dfrac{x}{6}+\dfrac{30}{6}\\
\LRA && 3x - 90	&= 4x - 60 - x - 30\\
\LRA && 3x - 4x + x	&= - 60 - 30 + 90\\
\LRA && 0x	&= 0\\
\end{aligned}$\\
Identieke vergelijking.
\end{enumerate}
\end{solution}

\newpage
\section*{Oplossingen A.C.O.}

\immediate\closeout\solutionstream
\input{\jobname.sol}

\end{document}
